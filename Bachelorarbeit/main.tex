\documentclass[
 paper=A4,pagesize=automedia,fontsize=12pt,
 BCOR=15mm,DIV=22,
 twoside,headinclude,footinclude=false,
 %fleqn,             % fleqn = linksbündige Ausrichtung von Formeln
 bibtotocnumbered,          % Literaturverz. im Inhaltsverz. eintragen
 liststotoc,                % Abbildungsverz. im Inhaltsverz. eintragen
 listsleft,                 % Abbildungsverz. an der längsten Nummer ausrichten
 pointlessnumbers,          % kein Punkt nach Überschriftsnummerierung
 cleardoublepage=empty      % Vakatseiten ohne Paginierung
]{scrbook}
\setlength\parindent{0em}

% Kodierung, Schrift und Sprache auswählen
\usepackage[utf8]{inputenc}
\usepackage[T1]{fontenc}
%\usepackage[ngerman]{babel}
% damit man Text aus dem PDF korrekt rauskopieren kann
\usepackage{cmap}
% Layout: Kopf-/Fußzeilen, anderthalbfacher Zeilenabstand
\usepackage{scrpage2} \pagestyle{scrheadings}
                      \clearscrheadfoot
                      \ihead{\headmark}\ohead{\pagemark}
                      \automark[subsection]{section}
                      \setheadsepline{0.5pt}
\usepackage{setspace} \onehalfspacing
\deffootnote{1em}{1em}{\textsuperscript{\thefootnotemark }}
% Grafiken, Tabellen, Mathematikumgebungen
\usepackage{graphicx}
\usepackage{tabularx}
\usepackage{amsmath,amsfonts,amssymb}
% Darstellung von Fließumgebungen
\usepackage{flafter,afterpage}
\usepackage[section]{placeins}
\usepackage[margin=8mm,font=small,labelfont=bf,format=plain]{caption}
\usepackage[margin=8mm,font=small,labelfont=bf,format=plain]{subcaption}

\numberwithin{equation}{section}
\numberwithin{figure}{section}
\numberwithin{table}{section}

%%%%%%%%%%%%%%%%%%%%%%%%%%%%%%%%%%%%%%%%%%%%%%%%%%%%%%%%%%%%%%%%%%%%%%%%%%%%%%%%
% Ab hier ist Platz für eigene Ergänzungen (Pakete, Befehle, etc.)

% Dieses Paket liefert den Blindtext, der als Platzhalter in den Beispieldateien steht.
% Das kannst Du also entfernen, wenn Du den Blindtext nicht mehr brauchst.
\usepackage{lipsum}
\usepackage{multirow}
\usepackage{subcaption}
\usepackage[toc,page]{appendix}
\usepackage[T1]{fontenc}
\usepackage{beramono}
\usepackage{listings}
\usepackage[usenames,dvipsnames]{xcolor}
\lstdefinelanguage{Julia}%
  {morekeywords={abstract,break,case,catch,const,continue,do,else,elseif,end,export,false,for,function,immutable,import,importall,if,in,macro,module,otherwise,quote,return,switch,true,try,type,typealias,using,while},%
   sensitive=true,%
   alsoother={$},%
   morecomment=[l]\#,%
   morecomment=[n]{\#=}{=\#},%
   morestring=[s]{"}{"},%
   morestring=[m]{'}{'},%
}[keywords,comments,strings]%

\lstset{%
    language         = Julia,
    basicstyle       = \ttfamily,
    keywordstyle     = \bfseries\color{blue},
    stringstyle      = \color{magenta},
    commentstyle     = \color{ForestGreen},
    showstringspaces = false,
}

\renewcommand\thesection{\arabic{section}}
\renewcommand{\thepage}{\arabic{page}}
\newtheorem{theorem}{Theorem}
\begin{document}


\frontmatter


% Titelpageseite
\begin{titlepage}
 \begin{tabularx}{\linewidth}{X}
  \includegraphics[width=6cm]{Graphics/TU_Logo_SW}                                                          \\\hline\hline

  \vspace{4.5em}

  \begin{singlespace}\begin{center}\bfseries\Huge

  Simulated annealing of a vertex lattice model implementation of reversible classical computation

  \end{center}\end{singlespace}

  \vspace{5.5em}

  \begin{singlespace}\begin{center}\large
  Bachelor-Arbeit                                                                                  \\ zur Erlangung des Hochschulgrades \\
  Bachelor of Science                                                                              \\
  im Bachelor-Studiengang Physik
  \end{center}\end{singlespace}\medskip

  \begin{center}vorgelegt von\end{center}
  \begin{center}
  {\large OSKAR PFEFFER}                                                                           \\ geboren am 20.05.1996 in Genua, Italien
  \end{center}\medskip

  \begin{singlespace}\begin{center}\large
  Institut für Festkörper und Werkstoffforschung                                                 \\
  Fachrichtung Physik                                                                              \\
  Fakultät Mathematik und Naturwissenschaften                                                     \\
  Technische Universität Dresden                                                                  \\ 2017
  \end{center}\end{singlespace}
 \end{tabularx}
\end{titlepage}


% Gutachterseite
\thispagestyle{empty}\vspace*{48em}

Eingereicht am xx.September 2017\vspace{1.5em}
\par{\large\begin{tabular}{ll}
 1. Gutachter: & Prof.~Dr.~Jeroen Van Den Brink \\
 2. Gutachter: & Prof.~Dr.~Andrei Ruckenstein \\
 \end{tabular}}


% Abstractseite
\newpage
\begin{center}\large\bfseries Summary\end{center}

Abstract \\
English: \\
This project will study the simulated annealing as a method for reaching the solution of classical computational problems encoded in the ground state of a novel class of planar vertex models with boundaries. It was recently shown that such vertex models provide a statistical mechanics representation of reversible circuits for universal classical computation. Unlike most other statistical mechanics descriptions of classical computation, the vertex models display trivial bulk thermodynamic behavior independent of the computation they represents. Thus,  the complexity of specific computations is reflected in the dynamics of a system's relaxation into its ground state. This project shall study classical simulated annealing as the simplest and most intuitive dynamics for relaxation into the ground state one can  impose on vertex models representing specific reversible classical computations. Our goal is (a) to assess the efficiency of simulated annealing in reaching the solution of typical classical computational problems;  and (b) to quantify the ``complexity'' of the computation as measured by the scaling of the ``time-to-solution'' with the size and difficulty of the computational problem.


\vspace{10em}
Abstract \\
Deutsch: \\
s

% Inhaltsverzeichnis

%\cleardoublepage
\tableofcontents

% Hauptteil
\pagenumbering{arabic}


\chapter{Introduction}

In recent years problems of computer science have gotten ever more attention from the physical community, which suggested new approaches connecting computational problems to physical systems, thus providing some insights from statistical physics \cite{Altarelli}. From simulated annealing (SA) over deep learning algorithms and other algorithms based on quantum mechanics or even quantum annealers, there is much hope to find more efficient ways to solve some interesting computational problems in the class NP and NP-complete, such as ground state finding and integer factorization \cite{Henelius}.

The motivation for this increased interest of the physical community for some complexity theory problems is that they can be mapped on physical systems and vice versa, thus delivering a broader understanding of both subjects and opening the possibility of new approaches that benefit from physical and computational insights.

This paper will focus on a recent work \cite{Chamon} on reversible classical computing, which provided a new tool that makes use of simulated annealing to find the result of a generic computation encoded in the ground state.
In their work they introduce a quantum vertex lattice and show some of its proprieties.
This paper will examine some further proprieties such as its complexity given by the efficiency of thermal annealing in reaching the solution of the system.

The first part of the paper will provide some theoretical background about complexity theory, classical annealing and the vertex model to be studied. In the second part, the simulations will be presented and finally in the last part the implications of the results on the proposed system will be discussed.




\chapter{Theoretical background}

\section{Classical annealing \label{sec:SA}}

Classical simulated annealing (SA), which was independently developed by \cite{Cerny} and \cite{Kirkpatrick2} is a ground state finding algorithm inspired by the typical annealing process in the metallurgy (heating followed by a controlled cooling down) and based on some central notions of statistical mechanics. Broad explanations of this topic can be found in \cite{SA} as well as in most other books on statistical mechanics algorithms.

The basic concept of annealing is to warm up the simulated system so that it can evolve almost free from any constraining potential and  let it cool down slowly, so that it can settle down in the lowest energy state, the same way a metal cools down and grows a single crystal if the temperature was lowered slowly enough.
For easy enough systems this indeed happens very quickly, since there usually are not many local minima and the energy difference between the local minima and the global minimum is generally big.
However, when the problem gets more complex the SA algorithms needs a longer cooling time to find the global minimum.
This is similar to the situation when in metallurgy the metal has some impurities and a fast cool down would not let the probe settle in the most ordered state.

Simulated annealing makes use of statistical mechanics and the properties of large number of particles. This can be described using the Gibbs ensemble, where the probability of a system to be in a state $\alpha$ with energy $E_\alpha$ is given by the Boltzmann-Gibbs distribution
\begin{equation}\label{eq:Boltzmann-Gibbs}
  P_\alpha = \frac{1}{Z}e^{\frac{-E_\alpha}{k_BT}} \text{,}
\end{equation}
where $k_B$ is the Boltzmann constant ($k_B=1$ for simulated annealing), T is the computational temperature of the system and Z is the partition function, given by
\begin{equation}\label{eq:partition_function}
  Z = \sum_\beta e^{\frac{-E_\beta}{k_BT}} \text{,}
\end{equation}
where the sum goes over all possible configurations of the system denoted by $\beta$.

Using this description it is possible to identify two different regimes.
The high temperature regime, where the `chaotic' states are the most probable states, since the system is free from any constraints and there are exponentially more unordered states than ordered states. And the more interesting low temperature regime, where the system settles down in a low energy state even though they are extremely rare, because they are preferred by the system. I.e. as T decreases eq. \eqref{eq:Boltzmann-Gibbs} collapses into only lowest energy states.
Take for example a ferromagnetic Ising spin chain with energy $E = -J \sum_{\langle i, j \rangle} \sigma_i \sigma_j$, where $\sigma = \pm 1$ and $J>0$.
At high temperature the spins will have a random order and the most probable state will have zero energy.
When the temperature is decreased the spins will align and close to zero temperature the most probable state is when all the spins are aligned with the ground state energy $E=-NJ$, where $N$ is the number of spins.

For practical applications it is not enough to require close to zero temperature, as it is possible that the system freezes  before it reaches the ground state due to local metastable states where it stays trapped.
This freezing can happen for example in the iterative improvement procedure, where only downhill moves in the energy are allowed until there is no possible move that decreases the energy. This procedure would let the system get stuck in a local minimum in the energy.

A better procedure is described by the Metropolis-Hastings (M-H) algorithm \cite{Metropolis} that simulates a system of particles in equilibrium at a given temperature.
When applied to an Ising spin model this algorithm visits each spin at random, inverts it (from $-1$ to $1$ and vice versa), and computes the change of the energy $\Delta E$. If $\Delta E \leq 0$ the inversion of the spin is accepted.
Otherwise, the change is accepted with probability $P_T(\Delta E) = e^{-\frac{\Delta E}{T}}$. This is usually implemented by generating a random number $\chi \in \left[ 0 , 1 \right]$ and using the accepting criterion $\chi < e^{-\frac{\Delta E}{T}}$.
This procedures mimics a system connected to a heat bath with temperature $T$ and ensures that the system will stochastically evolve into a Boltzmann-Gibbs distribution.

While it is possible to let the system reach equilibrium by holding the temperature very low, it is way more efficient to start at a higher temperature and slowly decrease it as it is done in the simulated annealing procedure, that describes a procedure where the system is constantly hold at equilibrium while the temperature is decreased.
Or, more clearly, the system starts at a high energy and trough the M-H algorithm it is allowed to reach equilibrium, then the temperature is slightly lowered and another step of the M-H is done so that the system stays in equilibrium.
Then the temperature is further lowered until it is zero and the ground state energy is reached.
The difficulty of the computation lies in the fact that the changes in temperature needed to preserve equilibrium sometimes become exponentially smaller while going to lower temperatures so that in practice it becomes impossible to perform an `adiabatic' simulated annealing.

The particular usefulness and efficiency of this the M-H algorithm, as opposed for example to iterative improvement, lies in the fact that it will surely converge to thermal equilibrium (thermalization) and therefore lead the system to the desired configuration.
This is ensured by the following theorem proven in \cite{Sethna}.
\begin{theorem}\label{th:equilibrium}
  A discrete dynamical system with a finite number of states can be guaranteed to converge to an equilibrium distribution $\rho$ if the computer algorithm
  \begin{itemize}
    \item is Markovian (has no memory),
    \item is ergodic (can reach everywhere and is acyclic), and
    \item satisfies detailed balance.
  \end{itemize}
\end{theorem}

It is easy to see that the M-H algorithm satisfies in principle these premises.
First of all, it represents a Markovian chain, since the configuration of the system $\chi^{n}$ depends only on the configuration right before it $\chi^{n-1}$, i.e. the acceptance probability $P_T(\Delta E)$ does not depend on past configurations and has therefore no memory.

The ergodicity of the system is given by the fact that there is a probability $p>0$ for the system to go from a state to any other state in a finite number of steps.
This propriety is mathematically obvious since any change can be accepted as given by $P_T(\Delta E)$, however we will see that for practical applications the algorithm might become non-ergodic at low temperatures for glassy systems, where the number of steps needed to reach any other state of the system tends to infinity and the equilibrium distribution cannot be reached.

Lastly, the fact that the M-H algorithm satisfies detailed balance and reversibility is the main principle of this algorithm and how this is assured can be seen explicitly in \cite{M-H_algorithm}.
Detailed balance means that given a function $p(x,y)$ representing the transition probability of the system in a state $x$ into a state $y$ and a probability density $\pi(x)$ then
\begin{equation}
  \pi(x)p(x,y)=\pi(y)p(y,x) \text{.}
\end{equation}
From a thermodynamical perspective the reversibility of the M-H is satisfied, the system in a state $x$ has Boltzmann probability density $\pi(x)=\frac{1}{Z}e^{-\frac{E(x)}{T}}$ and the probability in the M-H algorithm to accept a change leading to a higher energy state $y$ is $p(x,y) = e^{-\frac{E(y)-E(x)}{T}}$ therefore leading to the detailed balance equation
\begin{align}
  \pi(x)p(x,y) = \frac{1}{Z} e^{-\frac{E(x)}{T}} \cdot e^{-\frac{E(y)-E(x)}{T}}
  = \frac{1}{Z} e^{-\frac{E(y)}{T}} = \pi(y) = \pi(y) p(y,x) \text{,}
\end{align}
where the probability of a downhill (energy lowering) change is $p(y,x)=1$.

The simulated annealing algorithm can be modified and adapted for many different systems, there are only a few requirements needed to make use of it. First of all, there has to be a system described by a great amount of variables that can be randomly slightly variated (e.g. spin value, displacement). Secondly, an energy function that describes the change of energy after the variation, and lastly an annealing schedule for the temperature.



%
% In order to be able to make use of SA, the problem usually needs to be mapped onto an Ising spin model (i.e. where each spin has states $+1$ or $-1$) and a valid energy function has to be formulated.
% Then in statistical thermodynamics the probability of a system to be in a state $\alpha$ with energy $E_\alpha$ is given by the Boltzmann-Gibbs distribution
% \begin{equation}\label{eq:Boltzmann-Gibbs}
%   P_\alpha = \frac{1}{Z}e^{\frac{-E_\alpha}{k_BT}} \text{,}
% \end{equation}
% where $k_B$ is the Boltzmann constant ($k_B=1$ for simulated annealing), T is the computational temperature of the system and Z is the partition function, given by
% \begin{equation}\label{eq:partition_function}
%   Z = \sum_\beta e^{\frac{-E_\beta}{k_BT}} \text{.}
% \end{equation}
% At high temperature all the states have similar occurrence probability, but when T goes to zero only states with low energy are probable.
%
% Taking these concepts from thermodynamics the algorithm can be formulated as follows. First of all, the system is initiated in a random state, where all the spins are randomly chosen.
% This configuration also represents an infinite temperature, where all the spins are free.
% Then the temperature is slowly decreased and the spins are visited randomly according to the Metropolis-Hastings algorithm [1] N. Metropolis, A.W. Rosenbluth, M.N. Rosenbluth. A.H. Teller and E. Teller, J. Chem. Phys. 21 (1953) 1087-1092] and the spin is flipped (from $+1$ to $-1$ and vice versa) if the following conditions are true:
% \begin{itemize}
%   \item The energy of the system with the flipped spin is lower.
%   \item $\chi < e^{-\frac{\Delta E}{T}}$, where $\chi \in \left[0,1\right]$ is randomly generated, T is the temperature of the system and $\Delta E$ is the energy difference between the system before and after the spin flip.
% \end{itemize}
%
% After each step (one step consists in visiting $N$ spins randomly, where $N$ is the total number of the spins), the temperature is further decreased until it reaches the value 0.
% With this procedure the system can do a coarse search for a minimum in the beginning, where it is allowed to accept more changes that increase the energy and then do a finer search in the end when $T$ is low.
%
% This is the basic SA algorithm that can be slightly modified to fit a specific problem.

\section{Complexity theory}
In order to have a good basic understanding of complexity theory I recommend the reader to take a look to some literature such as \cite{Garey} that gives a good overview on the matter. However, for this paper a brief introduction to k-SAT and the complexity classes is sufficient as well as necessary.

K-SAT is one of the most important problems in complexity theory and was the first problem to be proven NP-Complete. K-SAT stands for the satisfiability problem of a formula in clausal normal form (CNF) of clause length k. A CNF formula is composed of boolean variables or their negation connected by ORs thus forming a clause, that is connected to other clauses by an AND. The k-SAT problem consists in figuring out whether such a formula is satisfiable by some specific configuration of variables. E.g. the 2-SAT formula $(a \vee b) \wedge (\neg a \vee b) \wedge (a \vee \neg b)$ is satisfiable with the configuration $a=\text{TRUE}, b=\text{TRUE}$, whereas $a \wedge \neg a$ is not satisfiable.

Whereas k-SAT problems with few variables and few clauses look pretty easy to solve, the difficulty increases drastically with the problem size and for the worst case scenario the fastest known algorithms are exponential in the problem size ($\tau \sim e^N$, where N represents the system size) making this problem not efficiently solvable. Furthermore, this problem was the first to be proven to be in the complexity class NP-Complete \cite{Cook}.
This class of problems is particularly important and interesting, since every problem in this class can be mapped in polynomial time on any other problem of the class NP (Non-deterministic polynomial time), that contains any problem with a solution that can be checked in polynomial time, or more colloquially all the interesting problems.
This means, that if an algorithm was found that solves a problem in NP-Complete efficiently, then any other problem in the NP class could be solved efficiently (in polynomial time) as well, meaning that the size of the class NP would be equal to the size of the class P, representing problems solvable in polynomial time ($\tau \sim N^C$, where C is a fixed constant). This is also known as the P vs NP problem.

Even though computational complexity theory is one of the central research topics for computer scientists, its study is also increasingly playing an important role in physics, since a lot of similarities between computational complexity and physical systems have been shown.
From phase transitions \cite{Monasson} and critical behavior in k-SAT  \cite{Kirkpatrick} to Ising spin glass models or atomic clusters whose ground state finding problem is proven to be in NP-Complete \cite{Barahona} or quantum entanglement as a measure for complexity \cite{Chamon_entropies}, computational complexity theory can deliver a broader understanding of some physical problems and it can benefit from approaches motivated by statistical mechanics.

As we will see in some results of this paper, the computational complexity of some physical systems is represented also in the difficulty of statistical mechanics derived algorithms, such as the M-H and the thereupon based simulated annealing ,in reaching the ground state, since the steps needed to thermalize and eventually reach the ground state grow exponentially with the system size.
This exponential growth of the steps can be represented by the loss of ergodicity of the M-H algorithm so that the premises of the theorem \ref{th:equilibrium} are not longer satisfied and the system cannot reach the equilibrium distribution.
Various papers \cite{Mauro} \cite{Carleo} \cite{Mackenzie} describe the loss of ergodicity in physical systems when the temperature is lowered, as these kind of systems, that have a `liquid' phase at high temperature where they show ergodicity and a so called `glassy' phase where the system gets stuck in a local metastable energy minimum with an almost unavoidable energy barrier, represent one of the few remaining unsolved and challenging issues in theoretical condensed matter \cite{Complex_behavior_of_glassy_systems}.

Since the proof by Barahona \cite{Barahona} extended later by \cite{Istrail} showing that any non-planar Ising model is in NP-Complete, research in statistical mechanics of spin glass models and complexity theory have become more strongly connected leading to a lot of approaches that try to map problems in NP to spin models \cite{Lucas} and make use of methods and heuristics from statistical mechanics.
This intuition that physics could help in solving efficiently computational problems has also led to the formulation of the vertex model discussed in the next section, which represents a classical, universal, reversible computation formed by logical gates and therefore belonging to the class NP-Complete.

Finally, while the P vs NP problem still stays unanswered and will probably stay so for still longer, it is also very interesting to find practical or efficient algorithms for some less-than-worst-case instances of NP problems or even average scenarios, that are most relevant in practical applications. These scenarios are examined in this paper, where the simulated annealing approach tries to deliver a fast practical solution to some instances of an NP-Complete problem.


\section{Vertex Model \label{sec:VM}}

The Vertex Model that will be analyzed in this section and whereupon the simulations in this paper are based was presented in a recent work by Chamon et al. \cite{Chamon}.

\begin{figure}[bhtp]
  \centering
  \includegraphics[width=0.7\textwidth]{Graphics/latticeplot.png}
  \caption{Random instance of the lattice model with TOFFOLI (grey), IDID (yellow), SWID (red), IDSW (blue), SWSW (green) gates. The lines represent the edges between the gates that connect the bits, where thicker lines represent 2-bit bonds and thinner lines 1-bit bonds. The vertical axes has periodic boundary conditions (i.e. the longer lines) and the computation has input to the left and output to the right. \label{fig:latticeplot}}
\end{figure}

The presented model represents a 2D vertex lattice implementation of a classical, reversible computation.
The main idea behind this implementation is that any Boolean function can be written using only 3-bit TOFFOLI gates \cite{Toffoli}.
Therefore, given a particular 2D arrangement of the TOFFOLI gates that do the computation, the plane can be first filled with multiple SWAP gates in order to bring distant bits closer to each other until they are adjacent without letting edges intersect and secondly with trivial IDENTITY gates in order to get a 2D plane filling lattice. Finally, the gates IDENTITY and SWAP are combined together to form the four 3-bit gates IDID, SWID, IDSW, SWSW, that are required to have a vertex lattice with only 3-bit gates.
An example of the final lattice can be seen in fig. \ref{fig:latticeplot}, where the input of the system is located on the left, the horizontal axes represents the computational time and the output is on the right.
The lattice has periodic boundary conditions on the vertical axes and the gates are connected by either two bit bonds or one bit bonds in the horizontal direction.

Once the computational lattice of gates is set up we need to map it onto a spin system and write the corresponding Hamiltonian, in order to transform it into a physical system.
First of all, the Boolean variables $x_i = (1+\sigma_i)/2$ used for the computation are trivially mapped on spins $\sigma_i \in \{ -1, 1 \}$ and the operations made by the gates are encoded using only two-bodies interactions making use of ancillary bits similarly to the way they are introduced in \cite{Biamonte}.
The energy function is then designed in such a way that correct outputs of the gate have the lowest energy and unsatisfying states have higher energies and are therefore unstable.

The Hamiltonian governing the gates will not be of any significance in this paper since the SA runs will update whole gates (i.e. it will update 6 spins at once) and the gate constraints will always be satisfied. For the interested reader the energy function of the gates can be found in the original paper by Chamon et al.

The edges (i.e. the lines connecting the gates in fig. \ref{fig:latticeplot}) that connect the bits from one gate to the other are transformed in ferromagnetic bonds with constant $K>0$ connecting the output bit of one gate to the input bit of the other gate. This leads to the term in the system Hamiltonian for the edges:

\begin{equation}
  H_{\text{Edges}} = - K \sum_{\langle i, j \rangle} \sigma _i \sigma _j \text{,}
\end{equation}
where $\sum_ {\langle i, j \rangle}$ stands for the sum over all the spins connected by edges at the boundaries of the gates.

Finally, to complete the mapping of the computation to a spin system we need to implement the boundary conditions, that in the computation are given by the input bits at the boundary. This is easily done by adding a strong magnetic field on the corresponding bit sites resulting in the Hamiltonian
\begin{equation}
  H_{\text{boundary}}= - \sum_{i\in \text{boundary}} h_i \sigma_i \text{,}
\end{equation}
where $h_i>0$ for $x_i = 1$, $h_i<0$ for $x_i = 0$ and $h_i=0$ if the bit is not fixed.

The resulting classical Hamiltonian is then given by
\begin{equation}
  H_C = \sum_g E_g^J  \left( \{ \sigma \} _g \right) - K \sum _{\langle i, j \rangle} \sigma_i \sigma_j - \sum_{i\in \text{boundary}} h_i \sigma_i \text{,}
\end{equation}
where the first sum is over all the gates g and $E_g^J \left( \{ \sigma \} _g \right)$ is the energy function of the gate which depends on the configuration of the spins of the gate $\{ \sigma \} _g$.

In \cite{Chamon} it is shown how this vertex model can be mapped on the Chimera architecture of the D-Wave machine by using a quantum Hamiltonian with quantum spins instead of the here presented classical Hamiltonian. Moreover, it is shown how the complexity of a ground-state computation cannot be revealed by thermodynamics alone, but can be seen in the dynamics of the system's relaxation into the ground state, which will be studied in this paper.


\chapter{Results}

In order to examine the dynamics and the behavior of the vertex model we studied many different cases of the random vertex lattice and used two different techniques to reach the ground state, \textit{`Vanilla' Annealing} and \textit{Annealing with Learning} \cite{Chamon}.
First we looked at the vertex model with open boundary (OB) conditions, meaning that no gate on the boundary is fixed and the system has $2^{3L}$ ground state configurations to choose from.
There we do a finite size energy scaling with the annealing time for the SA algorithm and show how the exponential behavior of the scaling varies with the concentration of Toffoli gates.
Then we looked at the fixed (FB) and mixed (MB) boundary conditions, where respectively the left boundary is completely fixed or $66\%$ of the gates are fixed on each boundary and there is a unique ground state where the system can settle.
In this case we show the difference between linear behavior for $0\%$ Toffoli and exponential behavior for $25\%$ Toffoli gates even for the `easy' case with a fixed boundary.
Finally, we show some proprieties of an improved protocol for the SA algorithm that leads to an exponential speedup for random squared systems with mixed boundary conditions.
The speedup of the enhanced Annealing with Learning makes this technique of possible practical relevance.

The plain code of the simulations is written in \textit{JULIA}, a high-level dynamic programming language, and can be found in the appendix.

\section{Thermal Annealing with open boundaries}
In this section we look at the dynamics of relaxation into the ground state for a system with open boundaries (OB) via `Vanilla' Annealing.
This is the standard SA algorithm described in section \ref{sec:SA}, where we start from a temperature of order $K$ down to zero over a total time $\tau$ following the temperature schedule $T(t) = K(1-\frac{t}{\tau + 1})$ (the nominator $\tau +1$ was chosen for easiness in writing the code and since $\tau >>1$, $T(\tau) \approx 0$ is sufficiently small as to not make any difference in the simulation.), where $K$ is the ferromagnetic coupling constant connecting the bits of adjacent gates.

In this simulation, as well as in the rest of the paper, the SA algorithm updates whole gates instead of only single spins, so that the gate constraints are always satisfied.
This way the SA algorithm chooses between 8 possible configurations of the gate, from all the input bits equal 0 to all the input bits equal 1.

\begin{figure}[hbtp]
  \begin{subfigure}{.5\textwidth}
    \centering
    \includegraphics[width=1.\linewidth]{Graphics/OB_E_vs_decades_0and15toff.pdf}
    \caption{Energy density $\frac{E}{N}$ of a system of size $L \times W$ after a total annealing time of $2^{\text{decade}}$. The energy density in the $0\%$ TOFFOLI case drops drastically while the $15\%$ TOFFOLI case shows its hardness in reaching the ground state as expected from a problem in NP. The thermal equilibrium energy for every system is 0, so that $(E-E_\text{thermal})/{N} =  {E}/{N}$.\label{fig:OB_E_vs_dec}}
  \end{subfigure}
  \begin{subfigure}{.5\textwidth}
    \centering
    \includegraphics[width=1.\linewidth]{Graphics/OB_E_vs_1overW_12decades_50toff.pdf}
    \caption{Example of the finite size scaling of the energy for a system with $100\%$ TOFFOLI concentration after an annealing time of 17 decades ($\tau=2^{17}$). The energy scales approximatively linearly with $1/W$, so that the expected energy density for a system of infinite size is the intersection of the fitted line with the y-axis.
    The multiple data points for some $W$ come from the different sizes in $L$.\label{fig:OB_E_vs_1overW}}
  \end{subfigure}
  \caption{\label{fig:OB_intro}}
\end{figure}

The analysis of the OB case was done for systems with fixed length in y direction $L\in \{ 5,10,20 \}$ and varying the length of the system in x direction $W$ for values between 50 and 300, in order to get an estimate of the energy density for a system of infinite length $W$ after an annealing time $\tau$.
An example of the extrapolation of the energy density after Annealing in $\tau=2^{17}$ steps for a system of infinite size $W$ can be seen in fig. \ref{fig:OB_E_vs_1overW}, where it is estimated by the intersection of the fitted line with the y-axis.

Extrapolating the finite size energy for multiple decades allows us to find out the scaling of the energy density with the annealing time.
The expected behavior of the annealing time needed to find the ground state is polynomial for a zero TOFFOLI concentration and else exponential.

\begin{figure}[hbtp]
  \centering
  \includegraphics[width=0.5\textwidth]{Graphics/OB_0TOFF_E_vs_1overtau}
  \caption{The energy density of systems without TOFFOLI gates decreases polynomially with the annealing time, as expected of annealing on a one dimensional Ising spins chain. This result is a good check that the code is working correctly.}
  \label{fig:OB_0TOFF_codecheck}
\end{figure}

Indeed, we found that the data correspond to this assumption.
Figure \ref{fig:OB_0TOFF_codecheck} shows the polynomial behavior of the energy density with the annealing time.
From $\ln (E/N) \sim c \ln(1/\tau)$ to get the constant $c_0$ follows $\tau \sim E^{-1/c_0}$, where $c_0=0.307\pm 0.004$.
This polynomial behavior is also a quality check for the code that it is working correctly.

For the systems with TOFFOLI gates, we found that the scaling of the annealing time is exponential in the inverse energy density.
The results are shown in figure \ref{fig:OB_TOFF}.
In order to get the scaling we made the ansatz $E/N = m/\ln(\tau)^c$.
Therefore, plotting the logarithm of this formula we can extract the coefficient $m$ and the power constant $c$ by fitting a line.
More specifically:
\begin{align}
  \ln \left(\frac{E}{N}\right) &= \ln \left(\frac{m}{\ln( \tau)^c} \right)\\
  &= \ln(m) + c \ln(\ln(\tau)) \text{,}
\end{align}
where $\ln(m)$ is the intersection of the fitted line with the y-axis and c is its slope.

Thus solving this formula for $\tau$ and by redefining $m^{1/c}=\eta$ we get
\begin{align}
  \tau \sim e^{\eta \left(E/N\right)^{-1/c}}.
\end{align}


\begin{figure}[hbtp]
  \begin{subfigure}{.5\textwidth}
    \includegraphics[width=1.\linewidth]{Graphics/OB_E_vs_1overdecadenew.pdf}
    \caption{This log-log plot shows the exponential time needed to reach zero energy. Decade stands for $\ln(\tau)$. The slope of each fitted line is $c$, while the intersection of each fitted line with the y-axis is $m$. For the concentrations $5\%$ and $10\%$ the smaller systems were not considered because of finite size effects, thus the big error bars.}
    \label{fig:OB_E_vs_1overdecade}
  \end{subfigure}
  \begin{subfigure}{.5\textwidth}
    \includegraphics[width=1.\linewidth]{Graphics/OB_constant_values.pdf}
    \caption{Values of the constants $\eta=m^{1/c}$ and $c$ for different TOFFOLI concentrations. As the concentration of TOFFOLI increases, the problem gets harder, i.e. $c$ decreases.}
    \label{fig:OB_constant_values}
  \end{subfigure}
  \caption{}
  \label{fig:OB_TOFF}
\end{figure}


\section{Thermal Annealing with mixed and fixed boundaries}

In this section we present the results of the standard SA algorithm methods for squared systems with mixed (MB) and fixed (FB) boundary conditions.
FB conditions means that all the gates on the left boundary of the system are fixed and are not considered in the SA algorithm, so that they are initialized in one state and cannot be changed.
The MB conditions case is when the left and the right boundary are partially, but not completely fixed.
For both boundary conditions we made sure that there is a unique ground state where the system can settle.
While this is trivially satisfied for FB conditions, it is harder to check for random systems with MB conditions and we had to look for systems that satisfy this requirement.
This was done by randomly initializing the full left boundary of the system, solving the system by direct computation, then fixing some of the gates on the left and the right boundaries and finally doing a full search over all the possible configurations of the unfixed bits on the boundaries and checking whether there was only one possible configuration leading to a solution.
Since this process involves doing a full search, the time required to check this condition scales as $2^{3\times \# \text{unfixed gates}}$ and the biggest system size that we were able to analyze was $21\times 21$ gates.

Interesting is the computational meaning of FB and MB conditions.
While FB represent a trivial computation, e.g. multiplication of two numbers, that can always be solved in polynomial time, MB conditions instances are a lot more interesting and represent way harder problems in NP that are worth studying.
One common example is factorizing, which can easily be written with this vertex lattice system.
Taking the multiplication system, instead of fixing the full left boundary, where the two factors are fixed and the bits of the product are initialized to 0, it is possible to fix the final product on the output (right boundary), the bits of the product on the input to 0 and all the ancillary bits needed to write a reversible computation to 0.
This way the system for multiplication is transformed in a factorizing system.

The systems that we analyzed were squared systems $L\times L$, where L is a multiple of 3 from 3 to 21, for TOFFOLI gates concentrations of $0\%$ and $25\%$.
For the MB conditions we fixed $66\%$ of each boundary.
We let each system anneal for a total time of $8$ decades up to $17$ decades.

\begin{figure}[hbtp]
  \centering
  \includegraphics[width=0.5\textwidth]{Graphics/annealingscaling_timetosolution.pdf}
  \caption{Averaged time to solution for systems with $0\%$ TOFFOLI gates. The scaling of the time to solution is linear with the system size for both MB and FB as expected.}
  \label{fig:annealingscaling_timetosolution}
\end{figure}

Again we did a quality check and found the scaling of the time to solution for the case with $0\%$ TOFFOLI gates and got the same result both for MB and FB, as should be expected since there is no computation being done and the information is just transported from one boundary to the other.
The result was that the time to solution scales as $\tau_\text{MB} \sim L^{3.76\pm 0.15}$ for MB and $\tau_\text{FB} \sim L^{3.68\pm 0.12}$ for FB.

For the systems with $25\%$ TOFFOLI gates, we found that there is no difference in the scaling of FB and MB.
Even though the FB case is much easier in terms of computational complexity, since it is solvable in polynomial time by direct computation, Simulated Annealing fails in solving it efficiently and the difficulty of this problem gets as bad as with MB conditions.

In order to analyze the data we defined an order parameter $m$ different from the energy, that measures the overlap of the final state of the system with the solution,
\begin{equation}
  m = \frac{8}{7} \left[\frac{1}{L^2}\sum_g \delta(q_g^\text{fin},q_g^\text{sol}) -\frac{1}{8}\right],
\end{equation}
where the sum goes over all the gates and $q_g^\text{fin}$ and $q_g^\text{sol}$ represent the state of the gate $g$ of the annealed system and of the solution.
The motivation for this order parameter is that even though the system may be at low energies, the distance from the true solution may be big.
One spin flip in the middle raises the energy from the ground state to the first excited state, but the error propagates and the system might be still far away from the solution.
The solution with which the order parameter is evaluated was saved in the beginning when the full search for the systems was done.
This order parameter becomes $m=1$ when the solution is reached and is $m=0$ when the system is in a disordered state, where 1 every 8 gates agree with the solution by chance.

\begin{figure}[hbtp]
  \begin{subfigure}{.5\textwidth}
    \includegraphics[width=1.\linewidth]{Graphics/annealinscaling_orderparameter_vs_LxL.pdf}
    \caption{The difference between systems with and without TOFFOLI is instantly visible, the harder systems (red and green) are far from solution while the easier ones (black and blue) are already solved.}
    \label{fig:annscal_ordpar_LxL}
  \end{subfigure}
  \begin{subfigure}{.5\textwidth}
    \includegraphics[width=1.\linewidth]{Graphics/annealingscaling_orderparameter_vs_annealingtime_MB.pdf}
    \caption{Simulated Annealing of systems with MB condition. The systems with FB look analogously. The annealing time $\tau$ is given in decades.}
    \label{fig:annscal_ordpar_anntime}
  \end{subfigure}
  \caption{SA of systems with $0\%$ and $25\%$ TOFFOLI. The systems without TOFFOLI reach the solution easily while both MB and FB cases with TOFFOLI gates take exponential time.}
  \label{fig:annealingscaling}
\end{figure}

In figure \ref{fig:annealingscaling} the difference in complexity between systems with and without TOFFOLI gates is evident.
When there are no TOFFOLI gates the system reaches solution efficiently in polynomial time with the size.
However in systems with TOFFOLI gates the Simulated Annealing algorithm cannot solve it efficiently, not even when one complete boundary is fixed and the solution could be extracted by simple direct computation.

This can be explained by the loss of ergodicity of the algorithm and the glassiness of the system.
There are many low energy states and once the system settles in one of those local minima it is very hard for the algorithm to escape, because it requires many uphill moves in the energy and the system becomes glassy.
In order to maintain ergodicity at lower temperatures, i.e. to make sure that any point in the configuration space can be reached from any other point, the time required by the algorithm increases exponentially with the system size.
Once the algorithm becomes less ergodic it is not assured to reach solution, since the requirements of the theorem presented in section \ref{sec:SA} are not satisfied.

The fact that SA fails even to solve systems with FB conditions suggests that it cannot be a practical algorithm for this type of problems and a new approach is required.
This leads us to the last result of this paper where the proprieties of an improved protocol of the normal `Vanilla' Annealing is shown.

\section{Annealing with Learning}

The Annealing with Learning protocol is an improvement of the normal SA where the algorithms performs multiple thermal annealing steps and learns some new information after each step.
We found that this protocol leads to an exponential speedup compared to normal SA for squared systems and represents therefore a possibly interesting technique for practical applications.

The algorithm works as follows: (1) we start by annealing $N_R$ replicas of the same system with an annealing time $\tau$ that is long enough for the correlation length to grow beyond some columns of gates so that the final state of those gates after annealing coincides with the solution with a probability $p\sim exp(-\lvert x \rvert / \xi)>\frac{1}{2}$, where $\xi$  is the correlation length and $x$ the distance over which the correlation length has to grow; (2) if a fraction of the replicas higher than a threshold $\alpha$ agree on the assignment of a particular gate then this gate is fixed in this assignment; (3) the annealing protocol is repeated independently $N_R$ times with the particularity that fixed gates are not visited by the Metropolis-Hastings algorithm; (4) this procedure is repeated multiple times until all gates are fixed.

The number of replicas $N_R$ needed to ensure a correct assignment of a gate with an error rate smaller than $\epsilon$ is shown in \cite{Chamon} and is given by $N_{R\epsilon} = \ln \left[ \frac{2p-1}{p} \frac{\epsilon}{L^2}\right]/ \ln \left[ 2p^{1-\alpha}(1-p)^\alpha \right]$, where $\alpha$ is the learning threshold and $p>\frac{1}{2}$ is the probability of the correct assignment of a gate for one replica.
In our simulations we used 500 replicas of independently annealed systems to ensure a small error rate.

While the improvement of this method for a system with FB conditions was already shown in the original paper, where the time to solution decreases from exponential to polynomial, we analyze the scaling of Annealing with Learning for more interesting, harder problems with MB conditions.
An example of how the algorithm works for systems with MB conditions is shown in figure \ref{fig:colorplot_example}.

\begin{figure}[hbtp]
  \centering
  \begin{subfigure}{0.18\textwidth}
    \includegraphics[width=1.\linewidth]{Graphics/annealing_examples/a100_21x21_0666fixed_9dec_91.png}
  \end{subfigure}
  \begin{subfigure}{0.18\textwidth}
    \includegraphics[width=1.\linewidth]{Graphics/annealing_examples/a100_21x21_0666fixed_9dec_910.png}
  \end{subfigure}
  \begin{subfigure}{0.18\textwidth}
    \includegraphics[width=1.\linewidth]{Graphics/annealing_examples/a100_21x21_0666fixed_9dec_920.png}
  \end{subfigure}
  \begin{subfigure}{0.18\textwidth}
    \includegraphics[width=1.\linewidth]{Graphics/annealing_examples/a100_21x21_0666fixed_9dec_930.png}
  \end{subfigure}
  \begin{subfigure}{0.18\textwidth}
    \includegraphics[width=1.\linewidth]{Graphics/annealing_examples/a100_21x21_0666fixed_9dec_940.png}
  \end{subfigure}

  \caption{Annealing with Learning of a system with size $21\times 21$ with $100\%$ TOFFOLI gates and with an annealing time of 9 decades for each learning step. Shown is the overlap of the state of each gate with the solution averaged over 500 replicas after 1,10,20,30 and 40 learning steps. White means that the gate is fixed in the correct state, dark means random state configuration far from solution. After 43 learning steps the system was solved.}
  \label{fig:colorplot_example}
\end{figure}

Before going over to the efficiency of this algorithm we want to outline one of its defects.
Since the dynamic correlation length in the SA algorithm grows only exponentially with the annealing time, the requirement $p\sim \exp(-\lvert x \rvert / \xi)>\frac{1}{2}$ is often not satisfied if the annealing time is not long enough (to expect a speedup we cannot wait until the system is already solved by pure annealing) and the algorithm makes mistakes.
In the most difficult system configurations $15\%$ of the systems made a mistake before reaching solution, but for all the other configurations of the parameters (TOFFOLI concentration, system size, annealing time), the error rate was always under $10\%$.
Even though this error rate is pretty high, the order parameter for systems that made a mistake could nonetheless grow close to 1 where almost all the gates of the system were in the right assignment except for some gates in a small region.
An example of this behavior can be seen in figure \ref{fig:colorplot_mistake}.

\begin{figure}[hbtp]
  \centering
  \begin{subfigure}{0.18\textwidth}
    \includegraphics[width=1.\linewidth]{Graphics/annealing_examples/Error/a100_21x21_0666fixed_12dec_71.png}
  \end{subfigure}
  \begin{subfigure}{0.18\textwidth}
    \includegraphics[width=1.\linewidth]{Graphics/annealing_examples/Error/a100_21x21_0666fixed_12dec_75.png}
  \end{subfigure}
  \begin{subfigure}{0.18\textwidth}
    \includegraphics[width=1.\linewidth]{Graphics/annealing_examples/Error/a100_21x21_0666fixed_12dec_79.png}
  \end{subfigure}
  \begin{subfigure}{0.18\textwidth}
    \includegraphics[width=1.\linewidth]{Graphics/annealing_examples/Error/a100_21x21_0666fixed_12dec_713.png}
  \end{subfigure}
  \begin{subfigure}{0.18\textwidth}
    \includegraphics[width=1.\linewidth]{Graphics/annealing_examples/Error/a100_21x21_0666fixed_12dec_716.png}
  \end{subfigure}
  \caption{Annealing with learning after 1,5,9,13 and 16 learning steps. Dark means far away from the correct assignment, white means fixed in the right state. Most of the system is correctly solved except for a small connected region (the top and the bottom gates are connected by periodic boundaries on the vertical axis) and finding the true solution for this instance is an easy problem since the complete right boundary is fixed.($100\%$ TOFFOLI gates, size: $21\times 21$ gates, annealing time: 12 decades)}
  \label{fig:colorplot_mistake}
\end{figure}

Since the mistakes usually happen far away from fixed gates and the algorithm usually fixes new gates close to a fixed region, a possible improvement could be a an extra heuristical condition for fixing a gate, where the gate is only fixed if it is close to some correct region.
This way no unwanted gates are fixed where no information about the correct solution from the rest of the system has arrived yet.
However, this implementation has not been tried in this work and is not guaranteed to remove this defect from this algorithm.

Putting this problem aside, we did a scaling of the order parameter $m$ with the learning steps and annealing time without taking into consideration that some systems did commit mistakes.
Figure \ref{fig:3-21_theta} shows the dependence of the averaged order parameter with the learning steps and the total annealing time for a system of size $18\times 18$ and with only TOFFOLI gates.
From this plot we see that the order parameter grows logarithmically with the annealing time for a fixed number of learning steps.
This means that the order parameter scales as $m(Z)\sim c(Z) \ln (\tau)$, $c(Z)$ is the slope of the fitted lines and it depends on the number of learning steps $Z$.
On the right plot it is shown how $c$ scales linearly with $Z$, i.e. $c=c_0+\theta Z$, where $c_0$ is the scaling constant for pure SA and $\theta$ is some coefficient determining the slope of the scaling.

Repeating the same scaling for all the system sizes we get the scaling of $\theta$ with $L$.
On the log-log plot of figure \ref{fig:3-21_nu} we can determine the scaling of emacs
$\tau \sim \exp(\frac{m L^{-\nu}}{\# \text{learning steps}})$


\chapter{Discussion}

\begin{thebibliography}{9}
  \bibitem{Altarelli}
  Altarelli F, Monasson R, Semerjian G, Zamponi F.
  \textit{A review of the Statistical Mechanics approach to Random Optimization Problems.} Handbook of Satisfiability (2009)

  \bibitem{Henelius}
  Henelius P, Girvin S.
  \textit{A statistical mechanics approach to the factorization problem.} arXiv:1102.1296

  \bibitem{Chamon}
  Chamon C, Mucciolo E R, Ruckenstein A E, Yang Z-C. \textit{Quantum vertex model for reversible classical computing.} Nature Communications 8 (2017)

  \bibitem{Cerny}
  Černý V. \textit{Thermodynamical approach to the traveling salesman problem: An efficient simulation algorithm.} Journal of Optimization Theory and Applications, Volume 45, Issue 1, pp 41–51 (1985)

  \bibitem{Kirkpatrick2}
  Kirkpatrick S, Gelatt C D Jr., Vecchi M P. \textit{Optimization by Simulated Annealing}. Science VOL 220 (1983)

  \bibitem{SA}
  van Laarhoven P J, Aarts E. \textit{Simulated Annealing: Theory and Applications.} Springer (1987)

  \bibitem{Metropolis}
  Metropolis N, Rosenbluth A W, Rosenbluth  M N, Teller A H, Teller E. \textit{Equation of State Calculations by Fast Computing Machines.} J. Chem. Phys. 21 (1953) 1087-1092

  \bibitem{Sethna}
  Sethna J P.
  \textit{Entropy, Order Parameters, and Complexity}. Oxford University Press (2006)

  \bibitem{M-H_algorithm}
  Chib S, Greenberg E. \textit{Understanding the Metropolis-Hastings Algorithm}.The American Statistician, Vol. 49, No. 4. (1995)

  \bibitem{Garey}
  Garey M R, Johnson D S. \textit{Computers and Intractability: A Guide to the Theory of NP-Completeness}. W. H. Freeman and Company (1979)

  \bibitem{Cook}
  Cook S. \textit{The Complexity of Theorem-Proving Procedures.} Proceedings of the 3rd Annual ACM Symposium on Theory of Computing (1971)

  \bibitem{Monasson}
  Monasson R, Zecchina R, Kirkpatrik S, Selman B, Troyansky L.
  \textit{Determining computational complexity from characteristic `phase transitions'}. Nature VOL 400 (1999)

  \bibitem{Kirkpatrick}
  Kirkpatrick S, Selman B. \textit{Critical Behavior in the Satisfiability of Random Boolean Expressions.} Science VOL 264 (1994)

  \bibitem{Barahona}
  Barahona, F. \textit{On the computational complexity of Ising spin glass models.} J. Phys. A: Math. Gen. 15 (1982)

  \bibitem{Chamon_entropies}
  Chamon C, Mucciolo E R. \textit{Rényi entropies as a measure of the complexity of counting problems}. J. of Statistical Mechanics: Theory and Experiment (2013)

  \bibitem{Mauro}
  Mauro J C, Smedskjaer M M. \textit{Statistical mechanics of glass}. Journal of Non-Crystalline Solids (2014)

  \bibitem{Carleo}
  Carleo G, Becca F, Schirò M, Fabrizio M.
  \textit{Localization and Glassy Dynamics Of
Many-Body Quantum Systems.} Scientific Reports 2, Article number: 243 (2012)

  \bibitem{Mackenzie}
  Mackenzie N D, Young A P. \textit{Lack of Ergodicity in the Infinite-Range Ising Spin-Glass}. Physical Review Letters (1982)

  \bibitem{Complex_behavior_of_glassy_systems}
  Rubí M, Pérez-Vincente C. \textit{Complex Behaviour
of Glassy Systems}. Springer (1997)

  \bibitem{Istrail}
  Istrail S.
  \textit{Statistical mechanics, three-dimensionality and NP-completeness. I. Universality of intractability for the partition function of the Ising model across non-planar lattices}. Proceedings of the Thirty-Second Annual ACM Symposium on Theory of Computing (2000)

  \bibitem{Lucas}
  Lucas A. \textit{Ising formulations of many NP problems}.Front. Phys. (2014)

  \bibitem{Toffoli}
  Toffoli T. \textit{Reversible Computing.} Proceedings of the 7th Colloquium on Automata, Languages and Programming (1980)

  \bibitem{Biamonte}
  Biamonte J D.
  \textit{Nonperturbative k-body to two-body commuting conversion Hamiltonians and embedding problem instances into Ising spins}. Phys. Rev. A VOL 77, 052331 (2008)

\end{thebibliography}



\clearpage

\begin{appendices}
  \chapter{Code}
  \section{Graph library}
  \lstinputlisting[language=Julia, basicstyle=\footnotesize]{Graphslib.jl}
  \section{Vertex Model and Annealing module}
  \lstinputlisting[language=Julia, basicstyle=\footnotesize]{annealing.jl}
\end{appendices}


\minisec{Acknowledgements}
I would like to thank Dr. Stefanos Kourtis for his constant support and help before and during the work, Prof. Andrei Ruckenstein for pushing me into doing this thesis on this topic, and Prof. Claudio Chamon, for introducing me to the world of physical research and for welcoming me in his research group.
Finally, I would like to thank my parents, whom I owe everything and who made my academic education possible.
% Erklärung
\clearpage
\thispagestyle{empty}
\minisec{Erklärung}\vspace*{1.5em}

Hiermit erkläre ich, dass ich diese Arbeit im Rahmen der Betreuung am Institut
für Festkörper und Werkstoffforschung ohne unzulässige Hilfe Dritter verfasst und alle Quellen als solche gekennzeichnet habe.

\vspace*{45em}

Oskar Pfeffer \par
Dresden, September 2017
\end{document}
